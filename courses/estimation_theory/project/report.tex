\documentclass[letterpaper,10pt]{article}

\usepackage{fancyhdr}
\usepackage{extramarks}
\usepackage{amsmath}
\usepackage{amsthm}
\usepackage{amsfonts}
\usepackage{tikz}
\usepackage[plain]{algorithm}
\usepackage{algpseudocode}
\usetikzlibrary{automata,positioning}

\usepackage{parskip} % Adds spacing between paragraphs
\usepackage{indentfirst} % indent the first paragraph of a section
\setlength{\parindent}{15pt} % Indent paragraphs
%\setlength\parindent{0pt}

%
% Basic Document Settings
%

\topmargin=-0.45in
\evensidemargin=0in
\oddsidemargin=0in
\textwidth=6.5in
\textheight=9.0in
\headsep=0.25in

\linespread{1.1}

\pagestyle{fancy}
\lhead{\hmwkAuthorName}
%\chead{\hmwkClass\ (\hmwkClassInstructor\ \hmwkClassTime): \hmwkTitle}
\rhead{\hmwkClass\ (\hmwkClassInstructor): \hmwkTitle}
%\rhead{\firstxmark}
%\lfoot{\lastxmark}
\cfoot{\projectname}
\rfoot{\thepage}

\renewcommand\headrulewidth{0.4pt}
\renewcommand\footrulewidth{0.4pt}

%
% Create Problem Sections
%

\newcommand{\enterProblemHeader}[1]{
    \nobreak\extramarks{}{Problem \arabic{#1} continued on next page\ldots}\nobreak{}
    \nobreak\extramarks{Problem \arabic{#1} (continued)}{Problem \arabic{#1} continued on next page\ldots}\nobreak{}
}

\newcommand{\exitProblemHeader}[1]{
    \nobreak\extramarks{Problem \arabic{#1} (continued)}{Problem \arabic{#1} continued on next page\ldots}\nobreak{}
    \stepcounter{#1}
    \nobreak\extramarks{Problem \arabic{#1}}{}\nobreak{}
}

\setcounter{secnumdepth}{1} % set section numbering
\renewcommand\thesection{\Roman{section}.} % set section numbering style

\newcounter{partCounter}
\newcounter{homeworkProblemCounter}
\setcounter{homeworkProblemCounter}{1}
\nobreak\extramarks{Problem \arabic{homeworkProblemCounter}}{}\nobreak{}

\newenvironment{homeworkProblem}{
    \section{Problem \arabic{homeworkProblemCounter}}
    \setcounter{partCounter}{1}
    \enterProblemHeader{homeworkProblemCounter}
}{
    \exitProblemHeader{homeworkProblemCounter}
}

%
% Homework Details
%   - Title
%   - Due date
%   - Class
%   - Section/Time
%   - Instructor
%   - Author
%

\newcommand{\projectname}{Adaptive Multi-level Nested Sampling for Wideband Channel Estimation}
\newcommand{\hmwkTitle}{Course Porject}
\newcommand{\hmwkDueDate}{February 12, 2014}
\newcommand{\hmwkClass}{Distributed Estimation (EE5369)}
%\newcommand{\hmwkClassTime}{}
\newcommand{\hmwkClassInstructor}{Dr. Ioannis Schizas}
\newcommand{\hmwkAuthorName}{Qiong Wu}

%
% Title Page
%

\title{
    \vspace{2in}
    \textmd{\textbf{\projectname}}\\
    \normalsize\vspace{0.3in}\large{\textit{\hmwkClass:\ \hmwkTitle}}\\
    %\normalsize\vspace{0.1in}\small{Due\ on\ \hmwkDueDate\ at 3:10pm}\\
    %\vspace{0.1in}\large{\textit{\hmwkClassInstructor\ \hmwkClassTime}}
    \vspace{0.1in}\large{\textit{Instructor:\ \hmwkClassInstructor}}
    \vspace{3in}
}

\author{\textbf{\hmwkAuthorName}}
\date{}

\renewcommand{\part}[1]{\textbf{\large Part \Alph{partCounter}}\stepcounter{partCounter}\\}

%
% Various Helper Commands
%

% Useful for algorithms
\newcommand{\alg}[1]{\textsc{\bfseries \footnotesize #1}}

% For derivatives
\newcommand{\deriv}[1]{\frac{\mathrm{d}}{\mathrm{d}x} (#1)}

% For partial derivatives
\newcommand{\pderiv}[2]{\frac{\partial}{\partial #1} (#2)}

% Integral dx
\newcommand{\dx}{\mathrm{d}x}

% Alias for the Solution section header
\newcommand{\solution}{\textbf{\large Solution}}

% Probability commands: Expectation, Variance, Covariance, Bias
\newcommand{\E}{\mathrm{E}}
\newcommand{\Var}{\mathrm{Var}}
\newcommand{\Cov}{\mathrm{Cov}}
\newcommand{\Bias}{\mathrm{Bias}}

\begin{document}

\maketitle

\pagebreak

\section{Introduction}

Channel estimation is a well-studied problem in the fileds of telecommunication and signal processing. It has become a popular topic again due to its importance implementation in the modern wireless communication system. Combination of Space-Time Block Code (STBC) with Orthogonal Frequency Division Multiplexing (OFDM) has the potential to approach the information theoretical capacity limit of Multiple Input Multiple Output (MIMO) channels \cite{Ganesan:2001}. However, at the receiver side, most space-time equalizer require the knowledge of the Channel State Information (CSI) to recover the transmitted data. This information is usually obtained through training the coherent Maximum-Likelihood (ML) receiver \cite{Larsson:2003}. The drawbacks of training-based approaches and differential schemes have motivated an increasing interest in the development of blind channel estimation algorithms for STBC systems. 

Despite the high performances of ML algorithm, their computational costs become prohibitive for high-order modulations. In the case of BPSK or QPSK constellations, the blind-ML detection can be simplified to a Boolean Quadratic Program (BQP) \cite{Ma:2006}. For more general settings, iterative procedure can be employed to avoid the computational complexity of the ML approach. These include the Cyclic ML \cite{Larsson:2003} and the Expectation-Maximisation (EM) \cite{Li:2001} algorithms. However, these iterative methods require a careful initialization of the channel and/or symbols. In particular, a poor initialization can strongly affect the Symbol-Error Rate (SER) performance. To avoid these drawbacks, several authors have investigated the use of sub-space \cite{Ammar:2007} or second-order statistics \cite{Shahbazpanahi:2005, Via:2008} approaches. However, excluding some specific low-rate codes, these approaches fail to extract the channel in a full-blind context \cite{Ammar:2007, Shahbazpanahi:2005, Via:2008}. Several approach have been proposed in literature to solve this problem, including the transmission of a short training sequence \cite{Ammar:2007} or the use of precoders \cite{Via:2008}. However, these semi-blind methods cannot be employed in a non-cooperative scenario since they require modification of the transmitter.

One solution to avoid these limitations is to exploit Higher-Order Statistics (HOS) \cite{Mendel:1991}, which could be approached by Independent Component Analysis (ICA) \cite{Hyvarinen:2001} that was originally developed for non-coded systems, and then extended to STBC communications \cite{Iglesias:2008}. Nevertheless, these algorithms were limited to a sub-class of Orthogonal STBCs and their extension to the general class of STBCs is far from trivial \cite{Iglesias:2008}. On the other hand, although there was other literature providing promising performance \cite{Choqueuse:2011}, it did not take into account the dramatically increased computational complexity as calculating HOS, which was equivalent to going back and forth between the trade-off of complexity and performance.

The dilemma above was my motivation to design the algorithm of channel estimation based on HOS via multi-level nested sampling, which had been proved useful in decreasing computational complexity while still maintaining statistical features of the original signal \cite{Pal:2010}. The basic idea of co-prime processing was to use Chinese reminder theorem with Bezout's identity to identify multiple frequencies from under-sampled sequences \cite{Xia:1999}. Vaidynanathan \cite{Vaidynanathan1:2011} further confined the rates of downsampling to be co-prime, and  provided concrete demonstrations for sampling region so that either co-prime or nested sampled points were able to calculate all of the second order derivatives. Furthermore, both sampling methods had already been used with HOS \cite{Pal:2012, Wu:2014}. However, the $2q$-order nested array was used for estimating the direction of arrival, while the co-prime sampling for channel estimation suffered from high estimation variance and slow convergence. Another similar approach worth mentioning was sparse ruling sampling \cite{Ariananda:2012} which was used to estimate power spectral density. Although it was able to combine with co-prime sampling \cite{Dominguez:2013}, the extension to calculate HOS is nontrivial.

Based on the previous work \cite{Wu:2014}, I will develop an algorithm performing channel estimation using adaptive multi-level nested sampling, which means the algorithm will adaptively changing the ``nested'' part for calculating HOS in accordance with the instantaneous signal-to-noise (SNR) ratio of the channel. Besides, its performance and complexity will be analyzed in different scenarios, and compared with benchmark algorithm provided in \cite{Choqueuse:2011}.

\pagebreak

\begin{thebibliography}{1}
    \setlength{\parskip}{2pt} % decreasing vertical spacing between refs

  \bibitem{Ganesan:2001} G. Ganesan and P. Stoica, ``Space-time block codes: a maximum SNR approach,'' \emph{IEEE Trans. on Information Theory}, vol. 47, no. 4, pp. 1650-1656, May 2001.

  \bibitem{Larsson:2003} E. Larsson, P. Stoica, and J. Li, ``Orthogonal space-time block codes: maximum likelihood detection for unknown channels and unstructured intereferences,'' \emph{IEEE Trans. on Signal Processing}, vol. 51, no. 2, pp. 362-372, 2003.

  \bibitem{Ma:2006} W. Ma, B. Vo, and P. Ching, ``Blind ML detection of orthogonal space-time block codes: efficient high-performance implementations,'' \emph{IEEE Trans. on Signal Processing}, vol. 54, no. 2, pp. 738-751, 2006.

  \bibitem{Li:2001} Y. Li, C. Georghiades, and G. Huang, ``Iterative maximum likelihood sequence estimation for space-time coded systems,'' \emph{IEEE Trans. on Communications}, vol. 49, no. 6, pp. 948-951, 2001.

  \bibitem{Ammar:2007} N. Ammar and Z. Ding, ``Blind channel identifiability for generic linear space-time block codes,'' \emph{IEEE Trans. on Signal Processing}, vol. 55, no. 1, pp. 202-217, 2007.

  \bibitem{Shahbazpanahi:2005} S. Shahbazpanahi, A. Gershman, and J. Manton, ``Closed form blind MIMO channel estimation for othogonal space-time codes,'' \emph{IEEE Trans. on Signal Processing}, vol. 53, no. 12, pp. 4506-4517, 2005.

  \bibitem{Via:2008} J. Via and I. Santamaria, ``Correlation matching approaches for blind OSTBC channel estimation,'' \emph{IEEE Trans. on Signal Processing}, vol. 56, no. 12, pp. 5950-5961, 2008.

  \bibitem{Mendel:1991} J. M. Mendel, ``Tutorial on higher-order statistics (spectra) in signal processing and system theory: theoretical results and some applications,'' in \emph{Proc. of IEEE}, val. 79, no. 3, pp. 278-305, March 1991.

  \bibitem{Hyvarinen:2001} A. Hyvarinen, J. Karunen, and E. Oja, \emph{Independent Component Analysis}, John Wiley and Sons, 2001.

  \bibitem{Iglesias:2008} H. Iglesias, J. Garcia-Naya, and A. Dapena, ``A blind channel estimation strategy for the 2x1 Alamouti system based on diagonalising 4th order cumulant matrices,'' in \emph{Proc. International Conf. Acoustic Speech Signal Process.}, pp. 3329-3332, Las Vegas, USA, Mar. 2008.

  \bibitem{Choqueuse:2011} V. Choqueuse, A. Mansour, and K. Yao, ``Blind channel estimation for STBC systems using higher-order statistics,'' \emph{IEEE Trans. on Wireless Comm.}, vol. 10, no. 2 , pp. 495-505, 2011.

  \bibitem{Pal:2010} P. Pal and P. P. Vaidyanathan, `` Nested array: A novel approach to array processing with enhanced degrees of freedom,'' \emph{IEEE Trans. on Signal Processing}, vol. 58, no. 8, pp. 4167-4181, August 2010.

\bibitem{Xia:1999}
  X. Xia, ``On estimation of multiple frequencies in undersampled complex valued waveforms,'' \emph{IEEE Trans. Signal Processing}, vol. 47, no. 12, pp. 3417-3419, Dec. 1999.

\bibitem{Vaidynanathan1:2011}
  P. P. Vaidyanathan and P. Pal, ``Sparse sensing with co-prime samplers and arrays,'' \emph{IEEE Trans. Signal Proc.}, vol. 59, pp. 573-586, Feb. 2011.

  \bibitem{Pal:2012} P. Pal and P. P. Vaidyanathan, ``Multiple level nested array: an efficient geometry for $2q$-th order cumulant based array processing,'' \emph{IEEE Trans. on Signal Process.}, vol. 60, no. 3, pp. 1253-1269, 2012.

  \bibitem{Wu:2014} Qiong Wu and Qilian Liang, ``Co-prime Sampling for Higher-Order Statistics with Application to LTE Channel Estimation,'' \emph{IEEE Conf. on Comm. (ICC)}, Syndney, Australia, June, 2014.

  \bibitem{Ariananda:2012} D. D. Ariananda and G. Leus, ``Compressive wideband power spectrum estimation,'' \emph{IEEE Trans. on Signal Processing}, vol. 60, no. 9, pp. 4775-4789, September 2012.

  \bibitem{Dominguez:2013} M. E. Dominguez-Jimenez and N. Gonzalez-Prelcic, ``A class of circular sparse rulers for compressive power spectrum estimation,'' \emph{Proc. of Euro. Signal Processing Conference (EUSIPCO)}, September 2013.

\end{thebibliography}

% \pagebreak
% 
% \begin{homeworkProblem}
%     Give an appropriate positive constant \(c\) such that \(f(n) \leq c \cdot
%     g(n)\) for all \(n > 1\).
% 
%     \begin{enumerate}
%         \item \(f(n) = n^2 + n + 1\), \(g(n) = 2n^3\)
%         \item \(f(n) = n\sqrt{n} + n^2\), \(g(n) = n^2\)
%         \item \(f(n) = n^2 - n + 1\), \(g(n) = n^2 / 2\)
%     \end{enumerate}
% 
%     \textbf{Solution}
% 
%     We solve each solution algebraically to determine a possible constant
%     \(c\).
%     \\
% 
%     \textbf{Part One}
% 
%     \[
%         \begin{split}
%             n^2 + n + 1 &=
%             \\
%             &\leq n^2 + n^2 + n^2
%             \\
%             &= 3n^2
%             \\
%             &\leq c \cdot 2n^3
%         \end{split}
%     \]
% 
%     Thus a valid \(c\) could be when \(c = 2\).
%     \\
% 
%     \textbf{Part Two}
% 
%     \[
%         \begin{split}
%             n^2 + n\sqrt{n} &=
%             \\
%             &= n^2 + n^{3/2}
%             \\
%             &\leq n^2 + n^{4/2}
%             \\
%             &= n^2 + n^2
%             \\
%             &= 2n^2
%             \\
%             &\leq c \cdot n^2
%         \end{split}
%     \]
% 
%     Thus a valid \(c\) is \(c = 2\).
%     \\
% 
%     \textbf{Part Three}
% 
%     \[
%         \begin{split}
%             n^2 - n + 1 &=
%             \\
%             &\leq n^2
%             \\
%             &\leq c \cdot n^2/2
%         \end{split}
%     \]
% 
%     Thus a valid \(c\) is \(c = 2\).
% 
% \end{homeworkProblem}
% 
% \pagebreak
% 
% \begin{homeworkProblem}
%     Let \(\Sigma = \{0, 1\}\). Construct a DFA \(A\) that recognizes the
%     language that consists of all binary numbers that can be divided by 5.
%     \\
% 
%     Let the state \(q_k\) indicate the remainder of \(k\) divided by 5. For
%     example, the remainder of 2 would correlate to state \(q_2\) because \(7
%     \mod 5 = 2\).
% 
%     \begin{figure}[here]
%         \centering
%         \begin{tikzpicture}[shorten >=1pt,node distance=2cm,on grid,auto]
%             \node[state, accepting, initial] (q_0)   {$q_0$};
%             \node[state] (q_1) [right=of q_0] {$q_1$};
%             \node[state] (q_2) [right=of q_1] {$q_2$};
%             \node[state] (q_3) [right=of q_2] {$q_3$};
%             \node[state] (q_4) [right=of q_3] {$q_4$};
%             \path[->]
%                 (q_0)
%                     edge [loop above] node {0} (q_0)
%                     edge node {1} (q_1)
%                 (q_1)
%                     edge node {0} (q_2)
%                     edge [bend right=-30] node {1} (q_3)
%                 (q_2)
%                     edge [bend left] node {1} (q_0)
%                     edge [bend right=-30] node {0} (q_4)
%                 (q_3)
%                     edge node {1} (q_2)
%                     edge [bend left] node {0} (q_1)
%                 (q_4)
%                     edge node {0} (q_3)
%                     edge [loop below] node {1} (q_4);
%         \end{tikzpicture}
%         \caption{DFA, \(A\), this is really beautiful, ya know?}
%         \label{fig:multiple5}
%     \end{figure}
% 
%     \textbf{Justification}
%     \\
% 
%     Take a given binary number, \(x\). Since there are only two inputs to our
%     state machine, \(x\) can either become \(x0\) or \(x1\). When a 0 comes
%     into the state machine, it is the same as taking the binary number and
%     multiplying it by two. When a 1 comes into the machine, it is the same as
%     multipying by two and adding one.
%     \\
% 
%     Using this knowledge, we can construct a transition table that tell us
%     where to go:
% 
%     \begin{table}[ht]
%         \centering
%         \begin{tabular}{c || c | c | c | c | c}
%             & \(x \mod 5 = 0\)
%             & \(x \mod 5 = 1\)
%             & \(x \mod 5 = 2\)
%             & \(x \mod 5 = 3\)
%             & \(x \mod 5 = 4\)
%             \\
%             \hline
%             \(x0\) & 0 & 2 & 4 & 1 & 3 \\
%             \(x1\) & 1 & 3 & 0 & 2 & 4 \\
%         \end{tabular}
%     \end{table}
% 
%     Therefore on state \(q_0\) or (\(x \mod 5 = 0\)), a transition line should
%     go to state \(q_0\) for the input 0 and a line should go to state \(q_1\)
%     for input 1. Continuing this gives us the Figure~\ref{fig:multiple5}.
% \end{homeworkProblem}
% 
% \begin{homeworkProblem}
%     Write part of \alg{Quick-Sort($list, start, end$)}
% 
%     \begin{algorithm}[]
%         \begin{algorithmic}[1]
%             \Function{Quick-Sort}{$list, start, end$}
%                 \If{$start \geq end$}
%                     \State{} \Return{}
%                 \EndIf{}
%                 \State{} $mid \gets \Call{Partition}{list, start, end}$
%                 \State{} \Call{Quick-Sort}{$list, start, mid - 1$}
%                 \State{} \Call{Quick-Sort}{$list, mid + 1, end$}
%             \EndFunction{}
%         \end{algorithmic}
%         \caption{Start of QuickSort}
%     \end{algorithm}
% \end{homeworkProblem}
% 
% \pagebreak
% 
% \begin{homeworkProblem}
%     Suppose we would like to fit a straight line through the origin, i.e.,
%     \(Y_i = \beta_1 x_i + e_i\) with \(i = 1, \ldots, n\), \(\E [e_i] = 0\),
%     and \(\Var [e_i] = \sigma^2_e\) and \(\Cov[e_i, e_j] = 0, \forall i \neq
%     j\).
%     \\
% 
%     \part
% 
%     Find the least squares esimator for \(\hat{\beta_1}\) for the slope
%     \(\beta_1\).
%     \\
% 
%     \solution
% 
%     To find the least squares estimator, we should minimize our Residual Sum
%     of Squares, RSS:
% 
%     \[
%         \begin{split}
%             RSS &= \sum_{i = 1}^{n} {(Y_i - \hat{Y_i})}^2
%             \\
%             &= \sum_{i = 1}^{n} {(Y_i - \hat{\beta_1} x_i)}^2
%         \end{split}
%     \]
% 
%     By taking the partial derivative in respect to \(\hat{\beta_1}\), we get:
% 
%     \[
%         \pderiv{
%             \hat{\beta_1}
%         }{RSS}
%         = -2 \sum_{i = 1}^{n} {x_i (Y_i - \hat{\beta_1} x_i)}
%         = 0
%     \]
% 
%     This gives us:
% 
%     \[
%         \begin{split}
%             \sum_{i = 1}^{n} {x_i (Y_i - \hat{\beta_1} x_i)}
%             &= \sum_{i = 1}^{n} {x_i Y_i} - \sum_{i = 1}^{n} \hat{\beta_1} x_i^2
%             \\
%             &= \sum_{i = 1}^{n} {x_i Y_i} - \hat{\beta_1}\sum_{i = 1}^{n} x_i^2
%         \end{split}
%     \]
% 
%     Solving for \(\hat{\beta_1}\) gives the final estimator for \(\beta_1\):
% 
%     \[
%         \begin{split}
%             \hat{\beta_1}
%             &= \frac{
%                 \sum {x_i Y_i}
%             }{
%                 \sum x_i^2
%             }
%         \end{split}
%     \]
% 
%     \pagebreak
% 
%     \part
% 
%     Calculate the bias and the variance for the estimated slope
%     \(\hat{\beta_1}\).
%     \\
% 
%     \solution
% 
%     For the bias, we need to calculate the expected value
%     \(\E[\hat{\beta_1}]\):
% 
%     \[
%         \begin{split}
%             \E[\hat{\beta_1}]
%             &= \E \left[ \frac{
%                 \sum {x_i Y_i}
%             }{
%                 \sum x_i^2
%             }\right]
%             \\
%             &= \frac{
%                 \sum {x_i \E[Y_i]}
%             }{
%                 \sum x_i^2
%             }
%             \\
%             &= \frac{
%                 \sum {x_i (\beta_1 x_i)}
%             }{
%                 \sum x_i^2
%             }
%             \\
%             &= \frac{
%                 \sum {x_i^2 \beta_1}
%             }{
%                 \sum x_i^2
%             }
%             \\
%             &= \beta_1 \frac{
%                 \sum {x_i^2 \beta_1}
%             }{
%                 \sum x_i^2
%             }
%             \\
%             &= \beta_1
%         \end{split}
%     \]
% 
%     Thus since our estimator's expected value is \(\beta_1\), we can conclude
%     that the bias of our estimator is 0.
%     \\
% 
%     For the variance:
% 
%     \[
%         \begin{split}
%             \Var[\hat{\beta_1}]
%             &= \Var \left[ \frac{
%                 \sum {x_i Y_i}
%             }{
%                 \sum x_i^2
%             }\right]
%             \\
%             &=
%             \frac{
%                 \sum {x_i^2}
%             }{
%                 \sum x_i^2 \sum x_i^2
%             } \Var[Y_i]
%             \\
%             &=
%             \frac{
%                 \sum {x_i^2}
%             }{
%                 \sum x_i^2 \sum x_i^2
%             } \Var[Y_i]
%             \\
%             &=
%             \frac{
%                 1
%             }{
%                 \sum x_i^2
%             } \Var[Y_i]
%             \\
%             &=
%             \frac{
%                 1
%             }{
%                 \sum x_i^2
%             } \sigma^2
%             \\
%             &=
%             \frac{
%                 \sigma^2
%             }{
%                 \sum x_i^2
%             }
%         \end{split}
%     \]
% 
% \end{homeworkProblem}
% 
% \pagebreak
% 
% \begin{homeworkProblem}
%     Prove a polynomial of degree \(k\), \(a_kn^k + a_{k - 1}n^{k - 1} + \hdots
%     + a_1n^1 + a_0n^0\) is a member of \(\Theta(n^k)\) where \(a_k \hdots a_0\)
%     are nonnegative constants.
% 
%     \begin{proof}
%         To prove that \(a_kn^k + a_{k - 1}n^{k - 1} + \hdots + a_1n^1 +
%         a_0n^0\), we must show the following:
% 
%         \[
%             \exists c_1 \exists c_2 \forall n \geq n_0,\ {c_1 \cdot g(n) \leq
%             f(n) \leq c_2 \cdot g(n)}
%         \]
% 
%         For the first inequality, it is easy to see that it holds because no
%         matter what the constants are, \(n^k \leq a_kn^k + a_{k - 1}n^{k - 1} +
%         \hdots + a_1n^1 + a_0n^0\) even if \(c_1 = 1\) and \(n_0 = 1\).  This
%         is because \(n^k \leq c_1 \cdot a_kn^k\) for any nonnegative constant,
%         \(c_1\) and \(a_k\).
%         \\
% 
%         Taking the second inequality, we prove it in the following way.
%         By summation, \(\sum\limits_{i=0}^k a_i\) will give us a new constant,
%         \(A\). By taking this value of \(A\), we can then do the following:
% 
%         \[
%             \begin{split}
%                 a_kn^k + a_{k - 1}n^{k - 1} + \hdots + a_1n^1 + a_0n^0 &=
%                 \\
%                 &\leq (a_k + a_{k - 1} \hdots a_1 + a_0) \cdot n^k
%                 \\
%                 &= A \cdot n^k
%                 \\
%                 &\leq c_2 \cdot n^k
%             \end{split}
%         \]
% 
%         where \(n_0 = 1\) and \(c_2 = A\). \(c_2\) is just a constant. Thus the
%         proof is complete.
%     \end{proof}
% \end{homeworkProblem}

\end{document}
