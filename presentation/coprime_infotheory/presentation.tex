%%%%%%%%%%%%%%%%%%%%%%%%%%%%%%%%%%%%%%%%%
% Beamer Presentation
% LaTeX Template
% Version 1.0 (10/11/12)
%
% This template has been downloaded from:
% http://www.LaTeXTemplates.com
%
% License:
% CC BY-NC-SA 3.0 (http://creativecommons.org/licenses/by-nc-sa/3.0/)
%
%%%%%%%%%%%%%%%%%%%%%%%%%%%%%%%%%%%%%%%%%

%----------------------------------------------------------------------------------------
%	PACKAGES AND THEMES
%----------------------------------------------------------------------------------------

\documentclass{beamer}

\mode<presentation> {

% The Beamer class comes with a number of default slide themes
% which change the colors and layouts of slides. Below this is a list
% of all the themes, uncomment each in turn to see what they look like.

%\usetheme{default}
%\usetheme{AnnArbor}
%\usetheme{Antibes}
%\usetheme{Bergen}
%\usetheme{Berkeley}
%\usetheme{Berlin}
%\usetheme{Boadilla}
\usetheme{CambridgeUS}
%\usetheme{Copenhagen}
%\usetheme{Darmstadt}
%\usetheme{Dresden}
%\usetheme{Frankfurt}
%\usetheme{Goettingen}
%\usetheme{Hannover}
%\usetheme{Ilmenau}
%\usetheme{JuanLesPins}
%\usetheme{Luebeck}
%\usetheme{Madrid}
%\usetheme{Malmoe}
%\usetheme{Marburg}
%\usetheme{Montpellier}
%\usetheme{PaloAlto}
%\usetheme{Pittsburgh}
%\usetheme{Rochester}
%\usetheme{Singapore}
%\usetheme{Szeged}
%\usetheme{Warsaw}

% As well as themes, the Beamer class has a number of color themes
% for any slide theme. Uncomment each of these in turn to see how it
% changes the colors of your current slide theme.

%\usecolortheme{albatross}
%\usecolortheme{beaver}
%\usecolortheme{beetle}
%\usecolortheme{crane}
%\usecolortheme{dolphin}
%\usecolortheme{dove}
%\usecolortheme{fly}
%\usecolortheme{lily}
%\usecolortheme{orchid}
%\usecolortheme{rose}
%\usecolortheme{seagull}
\usecolortheme{seahorse}
%\usecolortheme{whale}
%\usecolortheme{wolverine}

\setbeamertemplate{footline} % To remove the footer line in all slides uncomment this line
%\setbeamertemplate{footline}[page number] % To replace the footer line in all slides with a simple slide count uncomment this line

%\setbeamertemplate{navigation symbols}{} % To remove the navigation symbols from the bottom of all slides uncomment this line
}

\usepackage{graphicx} % Allows including images
\usepackage{booktabs} % Allows the use of \toprule, \midrule and \bottomrule in tables

%----------------------------------------------------------------------------------------
%	TITLE PAGE
%----------------------------------------------------------------------------------------

\title[Capacity Analysis of Coprime Communication]{Capacity Analysis of Coprime Communication} % The short title appears at the bottom of every slide, the full title is only on the title page

\author{Qiong Wu} % Your name
\institute[UCLA] % Your institution as it will appear on the bottom of every slide, may be shorthand to save space
{
University of Texas at Arlington \\ % Your institution for the title page
Wireless Comm. and Networking Lab\\
\medskip
\textit{qiong.wu@mavs.uta.edu} % Your email address
}
\date{\today} % Date, can be changed to a custom date

\begin{document}

\begin{frame}
\titlepage % Print the title page as the first slide
\end{frame}

\begin{frame}
\frametitle{Overview} % Table of contents slide, comment this block out to remove it
\tableofcontents % Throughout your presentation, if you choose to use \section{} and \subsection{} commands, these will automatically be printed on this slide as an overview of your presentation
\end{frame}

%----------------------------------------------------------------------------------------
%	PRESENTATION SLIDES
%----------------------------------------------------------------------------------------

%------------------------------------------------
\section{Introduction} % Sections can be created in order to organize your presentation into discrete blocks, all sections and subsections are automatically printed in the table of contents as an overview of the talk
%------------------------------------------------

%\subsection{Subsection Example} % A subsection can be created just before a set of slides with a common theme to further break down your presentation into chunks


\begin{frame}
\frametitle{Introduction}
The study of capacity of analog Gaussian channels and capacity-achieving transmission strategies was pioneered by Shannon, whose work focused on capacity of channels sampled at or above twice the channel bandwidth.\\~\\

However, in practice, the Nyquist rate may be {\bf excessive} for perfect reconstruction of signals that the transmission channel possess certain structures known a priori. On the other hand, the {\bf hardware} and {\bf power limitations} may preclude sampling at the Nyquist rate for a wideband communication system.\\~\\

This motivates the exploration of the effects of sub-Nyquist sampling upon the capacity of an analog Gaussian channel, and the capacity limits that result from general sampling methods.
\end{frame}

\begin{frame}
\frametitle{Related works}
\begin{itemize}
  \item When the channel or signal structure is unknown, the blind sub-Nyquist sampling approaches have be proposed to exploit the structure of various classes of input signals based on sampling with modulation and filter banks \cite{Mishali:2010}.
  \item At the transmitter side, although MRC maximizes the combiner SNR for a MISO channel, it is suboptimal for the joint optimization problem compared with selection combining \cite{Goldsmith:2005}.
  \item The capacity with sampling rate under modulation-bank are not monotonously increasing, which indicates that more sophisticated sampling techniques are necessary to maximize the capacity \cite{Chen:20132}.

\end{itemize}
\end{frame}

%------------------------------------------------

\section{Preliminaries}

\begin{frame}
\frametitle{Bullet Points}
\begin{itemize}
\item Lorem ipsum dolor sit amet, consectetur adipiscing elit
\item Aliquam blandit faucibus nisi, sit amet dapibus enim tempus eu
\item Nulla commodo, erat quis gravida posuere, elit lacus lobortis est, quis porttitor odio mauris at libero
\item Nam cursus est eget velit posuere pellentesque
\item Vestibulum faucibus velit a augue condimentum quis convallis nulla gravida
\end{itemize}
\end{frame}

%------------------------------------------------


\section{System Modeling}

\begin{frame}
\frametitle{Blocks of Highlighted Text}
\begin{block}{Block 1}
Lorem ipsum dolor sit amet, consectetur adipiscing elit. Integer lectus nisl, ultricies in feugiat rutrum, porttitor sit amet augue. Aliquam ut tortor mauris. Sed volutpat ante purus, quis accumsan dolor.
\end{block}

\begin{block}{Block 2}
Pellentesque sed tellus purus. Class aptent taciti sociosqu ad litora torquent per conubia nostra, per inceptos himenaeos. Vestibulum quis magna at risus dictum tempor eu vitae velit.
\end{block}

\begin{block}{Block 3}
Suspendisse tincidunt sagittis gravida. Curabitur condimentum, enim sed venenatis rutrum, ipsum neque consectetur orci, sed blandit justo nisi ac lacus.
\end{block}
\end{frame}

%------------------------------------------------

\section{Theorems and Proofs}

\begin{frame}
\frametitle{Multiple Columns}
\begin{columns}[c] % The "c" option specifies centered vertical alignment while the "t" option is used for top vertical alignment

\column{.45\textwidth} % Left column and width
\textbf{Heading}
\begin{enumerate}
\item Statement
\item Explanation
\item Example
\end{enumerate}

\column{.5\textwidth} % Right column and width
Lorem ipsum dolor sit amet, consectetur adipiscing elit. Integer lectus nisl, ultricies in feugiat rutrum, porttitor sit amet augue. Aliquam ut tortor mauris. Sed volutpat ante purus, quis accumsan dolor.

\end{columns}
\end{frame}

%------------------------------------------------
\section{Conclusions}
%------------------------------------------------

\begin{frame}
\frametitle{Table}
\begin{table}
\begin{tabular}{l l l}
\toprule
\textbf{Treatments} & \textbf{Response 1} & \textbf{Response 2}\\
\midrule
Treatment 1 & 0.0003262 & 0.562 \\
Treatment 2 & 0.0015681 & 0.910 \\
Treatment 3 & 0.0009271 & 0.296 \\
\bottomrule
\end{tabular}
\caption{Table caption}
\end{table}
\end{frame}

%------------------------------------------------

\begin{frame}
\frametitle{Theorem}
\begin{theorem}[Mass--energy equivalence]
$E = mc^2$
\end{theorem}
\end{frame}

%------------------------------------------------

\begin{frame}[fragile] % Need to use the fragile option when verbatim is used in the slide
\frametitle{Verbatim}
\begin{example}[Theorem Slide Code]
\begin{verbatim}
\begin{frame}
\frametitle{Theorem}
\begin{theorem}[Mass--energy equivalence]
$E = mc^2$
\end{theorem}
\end{frame}\end{verbatim}
\end{example}
\end{frame}

%------------------------------------------------

\begin{frame}
\frametitle{Figure}
Uncomment the code on this slide to include your own image from the same directory as the template .TeX file.
%\begin{figure}
%\includegraphics[width=0.8\linewidth]{test}
%\end{figure}
\end{frame}

%------------------------------------------------


\begin{frame}
\frametitle{References}
\scriptsize{
\begin{thebibliography}{1} % Beamer does not support BibTeX so references must be inserted manually as below

\bibitem{Mishali:2010}
    M. Mishali and Y. C. Eldar, ``From theory to practice: sub-Nyquist sampling of sparse wideband analog signals,'' \emph{IEEE J. Sel. Topics Signal Process.}, vol. 4, no. 2, pp. 375-391, Apr. 2010.

\bibitem{Goldsmith:2005}
    A. J. Goldsmith, \emph{Wireless Communications}. New York, USA: Cambridge Univ. Press, 2005.

\bibitem{Chen:20132}
    Y. Chen, Y. C. Eldar, and A. J. Goldsmith, ``Shannon meets Nyquist: capacity of sampled Gaussian channels,'' \emph{IEEE Transactions on Information Theory}, vol. 59, no. 8, pp. 4889-4914, Aug. 2013.

\bibitem{Chen:20131}
  Y. Chen, A. J. Goldsmith, and Y. C. Eldar, ``Channel capacity under sub-Nyquist nonuniform sampling,'' under revision, \emph{IEEE Transactions on Information Theory}, arXiv:1204.6049v4, 2013.

\bibitem{Szego:1984}
    U. Grenander and G. Szego, \emph{Toeplitz forms and their applications}, New York, NY, USA: Amer. Math. Soc., 1984.

\bibitem{Tilli:1998}
    P. Tilli, ``Singular values and eigenvalues of non-Hermitian block Toeplitz matrices,'' \emph{Linear Algebra Appl.}, vol. 272, no. 1–3, pp. 59-89, March 1998.

  \bibitem{Gray:2006}
      R. Gray, ``Toeplitz and circulant matrices: A review'', \emph{Foundations and Trends in Communications and Information Theory}, vol. 2, no. 3, pp. 155-239, Now Publishers, Delft, The Netherlands, 2006.

\end{thebibliography}
}
\end{frame}

%------------------------------------------------

\begin{frame}
\Huge{\centerline{Thanks!}}
\end{frame}

%----------------------------------------------------------------------------------------

\end{document} 
